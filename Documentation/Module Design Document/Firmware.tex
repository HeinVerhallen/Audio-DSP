\subsection{I2S decoder and encoder}
The ADC and DAC use I2S to transfer the sampled audio data to the FPGA. This protocol needs to be decoded to obtain the actual data, processed and be encoded again to I2S to send to the DAC. The VHDL code that has been written for this can be found in \nameref{chap:appendix-B-vhdl}. 

\subsection{I2C controller}
The DAC must be configured with I2C. To be able to do this an I2C master must be programmed in the FPGA. This I2C master initializes the DAC with default configurations. The VHDL code of the I2C master can be found in \nameref{chap:appendix-B-vhdl}. 

\subsection{Band-pass filter}
In order to successfully compute the state-space band-pass filter output, the discrete state-space coefficients are needed. The equations for the discrete coefficients are seen in equation \ref{eq:discrete-coef-Ad} and equation \ref{eq:discrete-coef-Bd}.

\begin{equation}
    Ad=e^{AT}
    \label{eq:discrete-coef-Ad}
\end{equation}

\begin{equation}
    Bd=\frac{Ad-I}{A}\cdot B
    \label{eq:discrete-coef-Bd}
\end{equation}

Because the coefficients are all matrices, doing an exponent computation is a very expensive task for an FPGA. Therefore the exponent is rewritten with the taylor series to obtain equation \ref{eq:taylor-series-Ad} and equation \ref{eq:taylor-series-Bd}. The higher the value of n in the sum for Ad and Bd the less of an effect that new value has on the sum. Therefore having the sum go from $n=0\ to\ n=10$ is more than enough for the coefficients to be almost exact. The coefficients need to be computed only at the startup of the firmware and when the parameters of the band-pass filter change. 

\begin{equation}
    Ad=\sum_{n=0}^{\infty}\frac{A^n\cdot T^n}{n!}
    \label{eq:taylor-series-Ad}
\end{equation}

\begin{equation}
    Bd=\sum_{n=1}^{\infty}\frac{A^{n-1}\cdot T^n}{n!}\cdot B
    \label{eq:taylor-series-Bd}
\end{equation}

When every sample is being loaded into the state-space band-pass filter effect, the new state variables and the output will be computed according to equation \ref{eq:state-state-equation} and equation \ref{eq:state-output-equation}. 

\begin{equation}
    \dot{x}=Ad \cdot x + Bd \cdot u
    \label{eq:state-state-equation}
\end{equation}

\begin{equation}
    y=C \cdot x + D \cdot u
    \label{eq:state-output-equation}
\end{equation}

After the computation of the state variables and the output, these data variables need to be resized to the desired size. This is necessary because when multiplying two binary numbers, the resulting numbers has the size equal to the sum of the size of the two multiplicands. In order to resize the values correctly a certain logic algorithm is used. This algorithm looks at the first bit that is high starting from the MSB. When the algorithm has found this bit, it will shift the value such that the found bit is now in the MSB position. Then it resizes the value to a given size where the MSB is kept but the LSBs are removed. 

The VHDL code of the state-space BPF can be found in \nameref{chap:appendix-B-vhdl}. 

\subsection{Sinewave generator}
In order to test the band-pass filter effect and the other effects, a sinewave generator is made to easily compare the input to the output of the effect. This makes the verification of the different effects very easy. The sinewave generator has an input that specifies the frequency of the generated sinewave. Because using an library that computes the sine for a certain value costs a lot of processing power, the taylor series is again used to approximate a sine. The taylor series of a sine is seen in equation \ref{eq:taylor-series-sin}. The VHDL code can be found in \nameref{chap:appendix-B-vhdl}. 

\begin{equation}
    \sin x = x - \frac{x^3}{3!} + \frac{x^5}{5!} - \frac{x^7}{7!} + \dots
    \label{eq:taylor-series-sin}
\end{equation}
