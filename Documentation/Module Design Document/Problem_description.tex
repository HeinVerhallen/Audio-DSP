\section{Background}
%BACKGROUND
When it comes to listening to music, it is crucial that the speakers are appropriately calibrated to both the surrounding environment and the position of the listener. This calibration is essential in order to achieve the best possible listening experience since sound pressure varies depending on the frequency at specific locations, characterized by nodes and antinodes. These nodes and antinodes shift throughout the space depending on the frequency. As a high fidelity (Hi-Fi) music listener, it is desirable to have the sound from all speakers reach you simultaneously. However, speakers are often not optimally positioned due to certain physical characteristics of the room. In cases where the speakers are not properly aligned with the surrounding environment, a digital signal processor (DSP) can be utilized to rectify this issue. A DSP is a specialized processor designed specifically for digital signal processing.

\noindent In the audio realm, DSPs are employed to optimize sound systems. Since perfect speakers do not exist, all speakers inherently possess certain imperfections. However, a DSP can compensate for these imperfections and provide corrective measures. Additionally, DSPs are frequently utilized to enhance the dynamics of sound and imbue it with a distinct character or sensation.

\noindent As a group, we have chosen to develop an audio DSP for the BeCreative minor because we are eager to learn how audio DSPs function and how to create one ourselves. Our ultimate goal is to be able to utilize this audio DSP to enhance the listening experience of music by correcting for speaker imperfections and applying specific presets. The system must meet the requirements outlined in the System Requirements Document (SRD), but what is most important to us is the opportunity for learning and acquiring valuable knowledge throughout the process.


\section{Problem description}

\section{Project goals}

\section{Project scope}

\section{Boundary condition}

\section{Project approach}

\section{Verification method}

