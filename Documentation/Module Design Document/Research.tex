\section{Research objectives}

To come up with a working DSP system, research should be conducted to understand and implement certain subjects into the end-product. This research varies from firmware to hardware, but also more specific about electronical systems and digital filters.

\section{Research questions}

\begin{itemize} %THIS IS TO MAKE LISTS
	\setlength\itemsep{-0.3em} %MAKES THE GAP SMALLER BETWEEN 2 ITEMS
	\item What is the best method for creating digital filters?
	\item What is the best method for creating digital effects?
	\item What is the most suitable anti-aliasing filter?
	\item What is the optimal needed roll-off for the anti-aliasing filter for a given bandwidth such that the noise can be negligible?
	\item What is the minimum sample frequency needed to capture the desired frequency spectrum?
	\item What is the minimum frequency range to be sampled to achieve sufficient detailed audio?
	\item What is the lowest allowable noise for decent audio?
	\item What analog to digital converter (ADC) resolution is needed such that the quantization error and noise level are on par?
	\item What ADC and digital to analog converter (DAC) architecture is most suitable for this application?
	\item What kind of processor is most suitable for this application?
	\item What is the permittable jitter for accurate audio?
	\item What is the maximum allowable ripple on the reference voltage for the ADC and DAC?
	\item How much RAM does the system need?
	\item How much flash does the system need?
	\item What power supply topology is best suited for each part of the system?
\end{itemize}

\section{Research approach}

By looking at the research questions and project requirements, research can be done to finalize this project. Research consists of looking at other DSP systems, dissecting certain parts of the main PCB how it should work, and making calculations on certain electronical parts.  
\par
\noindent This type of research is considered qualitative research. Data collection and mathematical calculations can together prove or disprove certain ideas on the final product or on certain parts of the project.
\par
\noindent
All the conducted research is available in the research document which can be found in Appendix G. This document contains all the gathered information (using a qualitative research method) needed to answer the research questions stated in paragraph 3.2. 

\section{Results}
\begin{itemize} %THIS IS TO MAKE LISTS
	\setlength\itemsep{-0.3em} %MAKES THE GAP SMALLER BETWEEN 2 ITEMS
		\item The results of the research is listed below:
		\item The best method for creating digital filters is by utilizing the fast fourier transform (FFT).
		\item The sample frequency should be as high as possible in order to achieve the optimal roll-off for the anti-aliasing filter. 
		\item The minimum sample frequency should be 40Khz.
		\item The minimum frequency range for high quality audio is 20Hz to 20KHz.
		\item The lowest allowable noise is a SNR of 90dB and for HiFI audio it is a SNR of 120dB.
		\item The resolution needed for the ADC and DAC is 24 bit.
		\item The best ADC type is the TI PCM18xx Series.
		\item The best DAC type is the TI PCM17xx Series.
		\item The most suitable processor is a FPGA.
		\item The amount of flash needed depends highly on the program size.
		\item The best power supply topology's are: linear voltage regulator, buck converter and sepic converter.
		\item The USB implementation costs too much time for this project.
\end{itemize} 