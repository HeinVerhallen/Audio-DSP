\subsection{User interface}\label{UI_Design}
The user interface(UI) is a critical part of the audio DSP project. The UI makes it that the user can interact with the system and set his desired settings.
What makes a good UI design is consistency, Readability and feedback. If every menu has the same style and general button layout to make sure menus are easy to navigate. Text in the UI should be kept to a minimum. This follows the saying that an image is worth a thousand words.If text is necessary it should be easy to read. The U should provide proper feedback, this means that if for example a page would take a second to load the UI should show that it is loading a page to prevent frustrations and possible errors.

\subsubsection{UI Components}
The UI will consist of the following three main components: A touchscreen, a rotary knob (with pushbutton) and a controller. The screen is a Nextion screen, meaning the menu design can be done in the Nextion editor.
The benefit of this is that we can design a menu in photoshop and lay transparent buttons on top of the parts that the user should be able to press.

The controller can control the screen visuals by sending commands via UART protocol. When an on screen button is pressed the Nextion screen sends can send the id of the button which is pressed. The menu navigation can be entirely done by using the rotary knob with pushbutton, but the screen also is a touchscreen so for simplicity's sake every on screen button can be used with touchscreen as well.

\subsubsection{UI Design}
The design was started with a diagram with all the necessary screens and information so that is was clear what menus should be in the DSP. There are 2 main menu layouts which are considered, namely function based and channel based. The function based layout is a layout where the functions are grouped and at the lowest level possible the channel has to be selected, for example in the function based layout the user would first navigate to the equalizer section and then select for which channel the equalizer has to be set.

The channel based layout is a layout where the user first selects the main functionality (signal routing, effects or presets) and after that selects the channel giving the user specific controls for the selected channel all in one place. Lets take a look at the previous example in the function based layout. In this case the user would navigate to the effect section, than select the channel to modify and lastly selecting the effect, in this case the equalizer.

\begin{figure}[ht]
    \includegraphics[width=\linewidth]{functionbasedUI}
    \caption{Function Based UI Design for a liquidCrystal screen}
    \label{fig:functionbasedUI}
\end{figure}

The channel based layout is chosen for the audio DSP UI, because it gives the user better overview of settings of the DSP, because all functions are sorted per channel.

\subsubsection{UI controller}

\begin{figure}[ht]
    \includegraphics[width=0.8\textwidth]{MainMenu1}
    \caption{Main menu design}
    \label{fig:mainmenu}
\end{figure}

\begin{figure}[ht]
    \includegraphics[width=0.8\textwidth]{adjustpreset1-2}
    \caption{Other menu screen, other ones are similar to this one}
    \label{fig:adjustpresetmenu}
\end{figure}


