\section{Method}
Time has not yet allowed for the creation of a neat and detailed test plan. However, before the voltage regulator boards are soldered onto the main board, they need to be thoroughly tested to verify their functionality. The plan is to create a test table for this purpose, similar to the specification of ICs in datasheets.

\subsection{Hardware}
\subsubsection{Voltage regulators}
For the voltage regulator modules, the following parameters need to be measured to demonstrate proper operation:
-Thorough visual inspection, especially focusing on the soldering process. This includes checking for short circuits, open connections, solder balls, and other defects.
-The converter should be powered from a lab power supply with low current limit settings. In case of a short circuit or other module defect, the dissipated power will be limited, allowing for the identification and resolution of the issue.
-The output voltage of the converter should be observed using an oscilloscope. Attention should be paid to the absolute accuracy and the amplitude of the ripple voltage.
-The temperature of the PCBA should be monitored to ensure that the module does not get too hot, which could potentially shorten its lifespan.

\subsubsection{Main board}
The functionality of the main board should be verified in the following manner:
-Perform a thorough visual inspection, with a focus on the soldering process. Check for short circuits, open connections, solder balls, and other defects.
-Power the main board from a lab power supply with low current limit settings. This will limit the dissipated power in case of a short circuit or module defect, enabling identification and resolution of the issue. A thermal camera can be used to inspect the PCBA for hotspots.
-Verify the presence and correctness of all power rails.
-Verify that the output of the GPIO pins corresponds to the desired outcome.(e.g. check the SLCK).
-Next, apply a known signal (e.g., a 1kHz sinusoidal waveform with an amplitude of 100mV) to the various inputs of the audio-DSP. If no filters are selected and the audio-DSP is essentially in a pass-through state, the same signal should appear on the selected outputs. The accuracy of this signal can be verified using an audio analyzer such as the AP SYS 2700.

\subsection{Firmware}
The firmware can be tested by simulating the VHDL code in Modelsim. For this .do files are made to force input signals. Due to time constrains only the I2S decoder and encoder have been tested with real audio. 

\subsubsection{I2S decoder and encoder}
For testing the I2S decoder and encoder, they are wired in series. So when a random data pattern is going in the decoder the encoder should output the same data pattern.

\subsubsection{I2C master}
The I2C master is tested by simply writing and reading data with the protocol and verifying that the protocol works correctly according to the I2C protocol standards. 

\subsubsection{State-space BPF}
The state-space BPF is tested by inputting a sine wave outputted from a sine wave generator. By using a sine wave as an input it is very easy to expect the outcome. For instance when having the BPF resonance frequency at 1kHz and inputting a 1kHz sine wave, it is expected that the output sine wave has the same amplitude compared to the input sine wave. And when the frequency of the input sine wave does not match that of the BPF resonance frequency, it will be attenuated. 

The best way to check if the band-pass filter works is by changing the frequency of the input sine wave over time to create a kind of frequency spectrum of the band-pass filter response. 

\subsubsection{Effects}
All the effects are tested by inputting a sine wave to the effect and looking at the result. This approach with a simple sine wave input is chosen because with a sine wave input it is very simple to predict the outcome of the effect and then confirm if it works as expected.

\section{Results}

\subsection{Hardware}

\subsubsection{Buck}
Due to the limited time, this verification has not been done yet.

\subsubsection{Linear voltage regulator}
After conducting a thorough visual inspection, it became apparent that a thorough cleaning of the PCBA was necessary. This was because the solder paste used left behind a significant amount of residue. Therefore, the linear voltage regulators were cleaned in an ultrasonic cleaner.

Subsequently, the linear regulators were connected to a lab power supply with current limiting. However, it quickly became evident that the regulators were not functioning properly. After performing several simulations in LTspice, it was suspected that the LEDs, which serve as low-noise voltage references, might have been soldered incorrectly. However, after consulting the datasheet again with several group members, it was confirmed that the LEDs were indeed soldered correctly according to the datasheet.

Since no other plausible causes emerged after further investigation, it was decided to desolder one of the LEDs and measure it out of circuit using a multimeter to determine its polarity and threshold voltage. To the astonishment of the group, it was discovered that the Broadcom datasheet for the HSMH-C170 LED did not match the LEDs actually supplied. The polarity marking in the datasheet was the exact opposite of the actual component. It is unknown whether an incorrect batch was delivered or if the datasheet itself is genuinely incorrect.
%https://nl.mouser.com/datasheet/2/678/AV02_0551EN_DS_HSMx_Cxxx_25Mar2022-1827675.pdf

\subsubsection{Main board}
\paragraph{inspection} Before connecting every module together and placing every component on the main board it is indeed ascertain that the board has flux and solder paste residue together with solder balls that can cause short circuits, therefore the PCBA will be cleaned in the ultrasonic bath, before doing this it is important to know which components are suitable for this. The relays and electrolytic capacitors are not suitable for this, therefore they are soldered after cleaning.  visual inspection has been done on the main board and the power delivery modules. In the second visual inspection there are no signs of shorts and the components are connected properly.
\paragraph{Visual inspection} 

Before connecting every module together and placing every component on the main board it is indeed ascertain that the board has flux and solder paste residue together with solder balls that can cause short circuits, therefore the PCBA will be cleaned in the ultrasonic bath, before doing this it is important to know which components are suitable for this. The relays and electrolytic capacitors are not suitable for this, therefore they are soldered after cleaning.  visual inspection has been done on the main board and the power delivery modules. In the second visual inspection there are no signs of shorts and the components are connected properly.

\paragraph{Vertical riser board}
The vertical riser boards have to be tested before placing it on the main board. These have been tested individually on ripple(not for the linear regulators, no equipment available to measure), output voltage and temperature. See \ref{Appendix-Vertical riser boards} Vertical riser boards in chapter \ref{chap:Appendix-Verification} verification for the outcomes. For the Sepic the output ripple is ~5mV and output is -15.11V which is within specs, the temperature is 70°C at 1A which is feasible but if there is more power required an extra fan can be attached to cool the converter more. The buck converters(see \ref{pdf:5V Buck converter} 5V buck - \ref{pdf:15V Buck converter} 15V buck) in the beginning all had an output of ~1v. This was because the feedback resistors were exchanged together. Therefore they had to be exchanged again and the problem is solved. The output voltage of all the buck converters are within spec. The ripple is ~8-10mV which is also within spec range. The temperature of all the buck converters are below 45°C at 1A. The output of the linear regulators (see \ref{pdf:12V linear regulator-1}12V lin reg up and till \ref{pdf:5V linear regulator-2}5V lin reg) are within specs and the temperature is 80°C at 1A, they won't be conducting more current than that so the temp is within spec range.

\paragraph{Main board power}
The PCBA is inspected and no short circuit is detected. The next step is to check all the power modules within the PCBA with current limiting, so that the board will not be catch on fire if there is somehow still a short. This has been done with a power supply limited to 1A. All the power rails are tested and every power delivery module is working properly at the desired voltage. The 12V buck converter that is going to the FPGA is tested separately first with four 33 resistors connected in parallel which will give 1.5A of current at 12V. The buck is still below 50°C. Therefore it is safe to power the fpga with the 12V buck converter

\paragraph{FPGA GPIO}
After doing visual inspection and vertical riser board inspection, it is safe to connect the fpga with the main board using flat cables. Unfortunately after connecting the the main board with the fpga it is stated that it didn't work after connecting the laptop using an RCA-cable to the audio dsp. The first thing that has been checked is the output of the clock going to the mainboard from the fpga. Doing so it has been identified that indeed the clock was not being transmitted correctly via the gpio pins of the DE1-SOC. In figure \ref{fig:Appendix-24.576MHz clock output} 24.576MHz clock output it is clearly visible that the 24.567MHz signal is clearly filtered due to parasitic capacitances in the signal path. Those effecting the paths are the following components Mosfets, diodes and the 47 ohm resistor at the GPIO pins. In figure \ref{fig:Appendix-50MHz clock output} 50MHz clock output is visible that the signal is even more weakened which makes it evident it is due to parasitic capacitances. In figure \ref{fig:Appendix-1MHz clock output} 1MHz clock output it is visible that the rising edge of the clock is already behaving capacitive. To be sure that the underlying issue is on all boards and not specifically on this board we have tested the clock output on another board and unfortunately the outcome is the same.

\subsection{Firmware}
\subsubsection{I2S encoder and decoder}
The results of this test can be found in figure \ref{fig:sim_result_i2s_dec_enc}. These results show that the I2S encoder and decoder work as expected. 

\subsubsection{I2C master}
The results can be seen in figure \ref{fig:sim_result_i2c-master}. In this result it can be seen that the I2C master works as expected according to the protocol standards.

\subsubsection{State-space BPF}

\subsubsection{Effects}
\paragraph{Volume control}

\section{Conclusions}

\subsection{Hardware}

\subsubsection{Buck}
Due to the limited time, this verification has not been done yet.

\subsubsection{SEPIC}
The SEPIC appears to be functioning well after some quick measurements. With an input voltage of +12V, it generates a stable -15V output voltage, and the voltage ripple remains nearly constant regardless of the output current. This aligns with the calculations and simulations. Additionally, the amplitude of the output ripple voltage closely matches the calculated and simulated values.

\subsubsection{Linear voltage regulator}
Due to the limited time, this verification has not yet been completed.

\subsubsection{Main board}
Due to the limited time, this verification has not been done yet.

\subsection{Firmware}
The I2S decoder, I2S encoder and I2C master work as expected. Therefore it is possible to sample audio using the ADC and DAC of the FPGA board. This has been tested and verified that this works. 

Due to time constrains further testing and implementation of the effects has not been conducted. 
