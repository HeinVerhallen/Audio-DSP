The goal of this project is to research how to make an audio-DSP. 
This raises the main research question: \textbf{“How to design an audio-DSP?”}. 
In the process of researching this an actual audio-DSP will be developed. 
From the main research question the following sub-research questions are derived:

\begin{itemize}
	\setlength\itemsep{-0.3em}
	\item What is the best method for creating digital filters?
	\item What is the best method for creating digital effects?
	\item What is the most suitable anti-aliasing filter?
	\item What is the optimal needed roll-off for the anti-aliasing filter for a given bandwidth such that the noise can be negligible?
	\item What is the minimum sample frequency needed to capture the desired frequency spectrum?
	\item What is the minimum frequency range to be sampled to achieve sufficient detailed audio?
	\item What is the lowest allowable noise for decent audio?
	\item What ADC resolution is needed such that the quantization error and noise level are on par?
	\item What ADC and DAC architecture is most suitable for this application?
	\item What kind of processor is most suitable for this application?
	\item What is the permittable jitter for accurate audio?
	\item What is the maximum allowable ripple on the reference voltage for the ADC and DAC?
	\item How much RAM does the system need?
	\item How much flash does the system need?
	\item What power supply topology is best suited for each part of the system?
	\item How do we ensure that the USB input of the processor is recognized by Windows as an audio device?
\end{itemize}

\noindent The project is conducted during the minor BeCreative at Fontys. 
This minor takes 20 weeks and allows the students to have a budget of €300,-. 
Thus after 20 weeks starting from 6-2-2023 an audio-DSP will be delivered within a budget of €300,-.

The audio system has some requirements to specify the final result. 
These requirements are derived with the “MoSCoW” method. 
The following requirements are confirmed by the research document.

\begin{longtable}{|c|p{10cm}|c|c|}
	\hline
	\textbf{ID} & \textbf{Requirement} & \textbf{Priority} & \textbf{Status}\\ \hline 
	\textbf{U1} & \textbf{Inputs:} \newline
	•Two RCA audio inputs which work on a line level of 4dBu(±1,74V)\newline
	•Two 6,35mm TRS plug audio inputs which work on a line level of 4dBu(±1,74)\newline
	•Two XLR audio which work on a line level of 22dBu(±9,75) 																	& Must	 & Proposed\\ \hline
	\textbf{U2} & \textbf{Outputs:} \newline
	•Two RCA audio outputs which work on a line level of 4dBu(±1,74V)\newline
	•Two XLR signal outputs which work on a line level of 22dBu(±9,75)
	 & Must & Proposed\\ \hline
	\textbf{U3} &The system should have a bandwidth (±3 dB) of at least 20 Hz up and till 20 kHz without any filters applied. 	& Must   & Proposed\\ \hline
	\textbf{U4} &The system has an Audio sample rate of at least 192 kHz 														& Must   & Proposed\\ \hline
	\textbf{U5} &The ADC and DAC resolution is at least 16-bit 																	& Must   & Proposed\\ \hline
	\textbf{U6} &Signal-to-noise and distortion (SINAD) is at least 100dB  														& Must   & Proposed\\ \hline
	\textbf{U7} &Anti-aliasing filter is a 6th order filter										 								& Must   & Proposed\\ \hline
	\textbf{U8} &propagation delay of less than 100ms without any filters applied												& Must   & Proposed\\ \hline
	\textbf{U9} &The system has two samplers											 										& Must   & Proposed\\ \hline
	\textbf{U10} &The system has two input samplers 																				& Must   & Proposed\\ \hline
	\textbf{U11}&The system has two output channels														 						& Must   & Proposed\\ \hline
	\textbf{U12}&The system has two signal processors														 					& Must   & Proposed\\ \hline
	\textbf{U13}&User can select what input will be routed to what channel via a user interface									& Must   & Proposed\\ \hline
	\textbf{U14}&User can select what output will be routed to what channel via a user interface								& Must   & Proposed\\ \hline
	\textbf{U15}&User can select 1 effect to be active in one channel at the same time											& Must   & Proposed\\ \hline
	\textbf{U16}&User can configure each effect															 						& Must   & Proposed\\ \hline
	\textbf{U17}&The system works standalone																					& Must   & Proposed\\ \hline
	\textbf{U18}&The user can configure each effect in the user interface														& Must   & Proposed\\ \hline
	\textbf{U19}&The in- and outputs can be soft-patched in the user interface													& Must   & Proposed\\ \hline
	\textbf{U20}&The system has a visual representation of the user interface													& Must   & Proposed\\ \hline
	\textbf{U21}&Effects configurable in each signal processor channel: \newline
	\begin{itemize}
		\setlength\itemsep{-0.4em}
		\item Distortion
		\item Reverb
		\item Gain
		\item Equalizer
		\item Delay
	\end{itemize}																												& Must 	 & Proposed\\ \hline
	\textbf{U22} &An FPGA is used as processor																 					& Must   & Proposed\\ \hline
	\textbf{U23} &RAM is at least 2MB																		 					& Must   & Proposed\\ \hline
	\textbf{U24} &The system should have a bandwidth (±1 dB) of at least 20 Hz up and till 20 kHz without any filters applied 	& Should & Proposed\\ \hline
	\textbf{U25} &The ADC and DAC resolution is at least 24-bit.																& Should & Proposed\\ \hline
	\textbf{U26} &Signal-to-noise and distortion (SINAD) is at least 120dB 														& Should & Proposed\\ \hline
	\textbf{U27} &The system has three samplers 																				& Should & Proposed\\ \hline
	\textbf{U28} &The system has three input channels 																			& Should & Proposed\\ \hline
	\textbf{U29} &The system has six output channels 																			& Should & Proposed\\ \hline
	\textbf{U30} &The system has six signal processors 																			& Should & Proposed\\ \hline
	\textbf{U31} &The system has a USB audio input					 															& Should & Proposed\\ \hline
	\textbf{U32} &Six XLR signal outputs work on a line level of 22 dBu (±9,75 V) 												& Should & Proposed\\ \hline
	\textbf{U33} &The system is able to recover the last saved configuration of the effect and the channel routing after reboot	& Should & Proposed\\ \hline
	\textbf{U34} &The system has equalizer presets e.g. Rock, Classical, Default, effect		 								& Should & Proposed\\ \hline
	\textbf{U35} &The system has different effect presets										 								& Should & Proposed\\ \hline
	\textbf{U36} &The system has default settings for channel routing and presets							 					& Should & Proposed\\ \hline
	\textbf{U37} &User can select up to 4 effects to be active in one channel at the same time. 								& Should & Proposed\\ \hline
	\textbf{U38} &Local power supplies for different parts of the system 														& Should & Proposed\\ \hline
	\textbf{U39} &Effects configurable in each signal processor channel:\newline
	\begin{itemize}
		\setlength\itemsep{-0.3em}
		\item Phaser
		\item Tremelo
		\item Flanger
		\item Fuzz
		\item Overdrive
		\item Chorus
		\item Compressor
		\item Wah
		\item Looper
		\item Wow and flutter
		\item Modulator
		\item Echo
		\item Fade in
		\item Delay (at least 4 seconds)
	\end{itemize}																												& Should & Proposed\\ \hline
	\textbf{U40} &Signal-to-noise and distortion (SINAD) is at least 140dB 														& Could  & Proposed\\ \hline
	\textbf{U41} &The user can configure custom presets for the equalizer, effect and channel routing via the user interface	& Could  & Proposed\\ \hline
	\textbf{U42} &User can select up to 10 effects to be active in one channel at the same time 								& Could  & Proposed\\ \hline
	\textbf{U43} &Touch screen user interface 																					& Could  & Proposed\\ \hline
	\textbf{U44} &Self-made mains power supply  																				& Won't  & Proposed\\ \hline
\end{longtable}